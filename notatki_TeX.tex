\documentclass{article}
\usepackage[a4paper,left=3.5cm,right=2.5cm,top=2.5cm,bottom=2.5cm]{geometry}
%%\usepackage[MeX]{polski}
%%\usepackage[cp1250]{inputenc}
\usepackage{polski}
\usepackage[utf8]{inputenc}
\usepackage[pdftex]{hyperref}
\usepackage{makeidx}
\usepackage[tableposition=top]{caption}
\usepackage{algorithmic}
\usepackage{graphicx}
\usepackage{enumerate}
\usepackage{multirow}
\usepackage{amsmath} %pakiet matematyczny
\usepackage{amssymb} %pakiet dodatkowych symboli
\begin{document}
%%	$ tu umie?? jaki? wz�z $ = \begin{equation} tu umiesc jaki? wz�r \end{equation}
	%% \ - oddzielenie
	%% Etykieta:
		%% Deklaracja: label{nazwa_etykiety}, 
		%% Odwo?anie: \equref{nazwa_etykiety}
	%% a nale??ce do zbioru A:
	%% a\in\mathbb{A}
	
	%% U?amek 1/x: \frac{1}{x}
	%% Index g�rny (pot?ga): ^{2}
	%% Index dolny: _2
	%% Pierwiastek z x: \sqrt{x}
	%% Pierwiastek stopnia y z x: \sqrt[y]{x}
	%% Nie r�wny: \neq
	%% Limes: lim_{}
	%% Znak niesko?czono?ci: \infty
	%% Inne znaczki:
\begin{equation}
	\sum \\
	\sum_{i=1}^{10}x_{i} \\
	\prod \\
	\coprod \\
	\int \\
	\oint \\
	\bigcap \\
	\bigcup \\
	\bigsqcup \\
	\bigvee \\
	\bigwedge \\
	\bigodot \\
	\bigotimes \\
	\bigoplus\\
	\biguplus \\
\end{equation}
\end{document}