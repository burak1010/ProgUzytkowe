\documentclass{article}
\usepackage[a4paper,left=3.5cm,right=2.5cm,top=2.5cm,bottom=2.5cm]{geometry}
%%\usepackage[MeX]{polski}
\usepackage[cp1250]{inputenc}
\usepackage{polski}
%%\usepackage[utf8]{inputenc}
\usepackage[pdftex]{hyperref}
\usepackage{makeidx}
\usepackage[tableposition=top]{caption}
\usepackage{algorithmic}
\usepackage{graphicx}
\usepackage{enumerate}
\usepackage{multirow}
\usepackage{amsmath} %pakiet matematyczny
\usepackage{amssymb} %pakiet dodatkowych symboli
\begin{document}
Tu umieszczamy kod TeXa, kt�ry bedzie kompilowany.

\tableofcontents
\section{tytul}
\subsection{podtytul}

\cite{dote:ks} etykieta

\begin{itemize}
	\item pierwszy punkt
	\item drugi 
	\item trzeci
	
	\begin{itemize}
		\item pierwszy podpunkt
		\item drugi
\end{itemize}
	
	\begin{enumerate}[I)]
		\item pierwszy 
		\item drugi
	\end{enumerate}
	
\end{itemize}

	\begin{enumerate}
		\item pierwszy 
		\item drugi
	\end{enumerate}

\begin{description}
	\item [itemize] wypunktowanie
	\item [enumerate] lista numerowana
\end{description}

\begin{thebibliography}{4}
	\bibitem{dote:ks} Donald E. Knuth, \TeX Przewodnik u�ytkownika, WNT, Warszawa
\end{thebibliography}

\end{document}