\documentclass{article}
\usepackage[a4paper,left=3.5cm,right=2.5cm,top=2.5cm,bottom=2.5cm]{geometry}
%%\usepackage[MeX]{polski}
\usepackage[cp1250]{inputenc}
\usepackage{polski}
%%\usepackage[utf8]{inputenc}
\usepackage[pdftex]{hyperref}
\usepackage{makeidx}
\usepackage[tableposition=top]{caption}
\usepackage{algorithmic}
\usepackage{graphicx}
\usepackage{enumerate}
\usepackage{multirow}
\usepackage{amsmath} %pakiet matematyczny
\usepackage{amssymb} %pakiet dodatkowych symboli
\begin{document}
%% Zadania do zrobienia : 
%% Tabele: 2(tabela 11), 3(bez numeru), 9(tabela 15), jedna kolorowa
%% Rysunki: 5, 6, 7
\begin{table}
%%\rowcolors{}
	\centering
		\begin{tabular}{|c|c|c|c|c|c|c|}
			\hline
			\multirow{2}{*}{No. of visual words} & \multicolumn{6}{c|}{Datset} \\ \cline{2-7} 
			  1 & 2 & 3 & 4 & 5 & 6 &  \\ \hline
			50 & 61.27\% & 88.92\% & 77.88\% & 87.89\% & 92.04\% & 96.95\% \\
			\hline
		\end{tabular}
	\caption{jakis tytul}
	\label{tab:dane}
\end{table}

\end{document}