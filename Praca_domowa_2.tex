\documentclass{article}
\usepackage[a4paper,left=3.5cm,right=2.5cm,top=2.5cm,bottom=2.5cm]{geometry}
%%\usepackage[MeX]{polski}
\usepackage[cp1250]{inputenc}
\usepackage{polski}
%%\usepackage[utf8]{inputenc}
\usepackage[pdftex]{hyperref}
\usepackage{makeidx}
\usepackage[tableposition=top]{caption}
\usepackage{algorithmic}
\usepackage{graphicx}
\usepackage{enumerate}
\usepackage{multirow}
\usepackage{amsmath} %pakiet matematyczny
\usepackage{amssymb} %pakiet dodatkowych symboli
\begin{document}
	\begin{table}
		\centering
			\begin{tabular}{c|c|c}
				\hline\hline
					$x_{1}$ & $x_{2}$ & $( x_{1} \text{AND} x_{2} )$ \\ \hline
					1 & 1 & 1 \\
					1 & 0 & 0 \\
					0 & 1 & 0 \\
					0 & 0 & 0 \\
					\hline	
					\hline
			\end{tabular}
			\caption{Praca Domowa: Tabela 11}
			\label{tab: 11}
	\end{table}
	
		\begin{table}
		\centering
			\begin{tabular}{|r|l|}
				\hline
					7C0 & hexadecimal \\
					3700 & octal \\ \cline{2-2}
					11111000000 & binary \\
					\hline \hline
					1984 & decimal \\
					\hline
			\end{tabular}
			\caption{Praca Domowa: Tabela bez podpisu}
			\label{tab: bez}
	\end{table}
	
	\begin{table}
		\centering
			\begin{tabular}{|r|l|}
			%%{|c|c|c|c|c|c|c|c|c}
					  %%\multicolumn{6}{|c|}{Primes} \\ \hline
					   %%& 2 & 3 & 5 & 7 & \\ \hline
						%%\multi{6}{4}{*}{Powers} \\
						%%\multirow{2}{*}{504} & 3 & 2 & 0 & 1 & \\
					\multicolumn{4}{c}{Primes} \\
					2 & 3 & 5 & 7  \\ 				
			\end{tabular}
			\caption{Praca Domowa: Tabela 15}
			\label{tab: 15}
	\end{table}
\end{document}