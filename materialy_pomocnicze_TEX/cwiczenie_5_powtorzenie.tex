\documentclass{article}
\usepackage[a4paper,left=3.5cm,right=2.5cm,top=2.5cm,bottom=2.5cm]{geometry}
\usepackage[MeX]{polski}
%\usepackage[cp1250]{inputenc}
\usepackage{polski}
%\usepackage[utf8]{inputenc}
\usepackage[pdftex]{hyperref}
\usepackage{makeidx}
\usepackage[tableposition=top]{caption}
\usepackage{algorithmic}
\usepackage{graphicx}
\usepackage{enumerate}
\usepackage{multirow}
\usepackage{amsmath} %pakiet matematyczny
\usepackage{amssymb} %pakiet dodatkowych symboli
\begin{document}
	\centering
		\textsc{Zespol Piersi} \\
		\LARGE{\textbf{Balkanica}}
		\\
		\normalsize{
	\begin{itemize}
		\item Balkanska w zylach plynie krew
		\item Kobiety, wino, taniec, spiew
		\item Zasady proste w zyciu mam
		\item Nie rob drugiemu tego czego ty nie chcesz sam!
		\item Lopa!
	\end{itemize}
	\begin{enumerate}
		\item Muzyka, przyjazn, radosc, smiech
		\item Zycie latwiejsze staje sie
		\item Przyniescie dla mnie wina dzban
		\item Potem ruszamy razem w tan
	\end{enumerate}}
		\centering
			Bedzie, bedzie zabawa! \\
			Bedzie sie dzialo! \\
			I znowu nocy bedzie malo. \\
		\begin{flushright}
			\textit{Bedzie glosno, bedzie radosnie \\ Znow przetanczymy razem cala noc \\ Lopa! Hej$!^{1}$}!\\
		 \end{flushright}
	\begin{flushleft}
	%\cline{5}
	\rule{6cm}{0.4pt}\\
	 \hspace{5pt} \footnotesize{$ ^{1}$Piersi - Bałkanica, autor tekstu: Adam Asanov, kompozytorzy: Adam Asanov i Zbigniew Moździerski, źródło: https://teksciory.interia.pl, dostęp: 23.03.2017}
	\end{flushleft}
	\begin{equation}
		|U|=sup\{|x-y|:x,y \in U\}
	\label{eq: 1}
	\end{equation}
	\begin{equation}
		F \subset \sum_{i=1}^{\infty} U_{i} \text{, gdzie } 0 <|U_{i}|\leqslant \delta \text{ dla ka�dego i,}
	\label{eq: 2}
	\end{equation}
	\begin{equation}
		H^8_{\delta} = inf \Bigg\{ \sum_{i=1}^{\infty}|U_i|^{s} :\{U_i\} \text{ jest } \delta \text{ - pokryciem F} \Bigg\}
	\label{eq: 3}
	\end{equation}
	\begin{equation}
		H^{s}(F) = \lim_{\delta\to 0} H_\delta^{s}(F)
	\label{eq: 4}
	\end{equation}
	\begin{equation}
		H^{s}(F) =  
			\left\{
				\begin{array}[c]
					c{\infty} \text{ je�li }s < dim_H F\\
					0 \text{ je�li } s > dim_H F \\
				\end{array}
			\right.
	\label{eq: 5}
	\end{equation}
	\begin{equation}
		\overline{dim}_B F = lim_{\delta \to 0} sup \frac{log N_\delta (F)}{-log \delta}
	\label{eq: 6}
	\end{equation}
	\begin{equation}
		\begin{array}[c]
			c{\overline{dim}_B F = \lim_{\delta \to 0} sup \frac{log N_\delta (F)}{-log \delta} \leqslant \lim_{k \to \infty} sup \frac{log2^{k}}{log3^{k-1}} = \frac{log2}{log3}} \\ \\
			\phi_t[\phi_\tau (x)] = \phi_{t+\tau} (x) \\ \\
			\frac{dx}{dt} = -10x + 10y \\
			\frac{dy}{dt} = rx - y - xz \\
			\frac{dz}{dt} = \frac{8}{3}z + xy	\hspace{16.5pt}
		\end{array}
	\label{eq: 7}
	\end{equation}
	\begin{equation}
		\begin{array}[c]
			c{\rho(f(x_1), f(x_2)) \leqslant \lambda \rho(x_1, x_2)} \\ \\
			R_U (\delta) = 
			\left[ 
					\begin{array}[c]
						r{cos\delta \ sin\delta \ 0} \\
						-sin\delta \ cos\delta \ 0 \\
						0 \hspace{15pt} \ 0 \hspace{10pt} \  1
					\end{array}
			\right]
		\end{array}	
	\label{eq: 8}
	\end{equation}
	
	\begin{algorithmic}
			\STATE{Algorytm zachlanny}\\
			\STATE{\hspace{10pt}\textsc{Greedy}(M,w)}\\
			\STATE{\hspace{10pt} uporzadkuj S[M] nierosnaco wedlug wagi w}
			\FOR{x$\in$ S[M], brane w porzadku nierosnacym wedlug wagi w(x)}
				\IF{$A \cup x \in \phi[M]$}
					\item{$A\leftarrow A \cup x$}
				\ENDIF
			\ENDFOR
			\RETURN{A}
			\\
			\STATE{\hspace{5pt} Algorytm znajdowania minimum}
			\STATE{\textsc{Minimum}'(A)}
			\FOR{$i\leftarrow 2$ to $lenght[A]\ $}
				\IF{$ min > A[i]\ $}
					\item $min\ \leftarrow A[i]$
				\ENDIF
			\ENDFOR
			\RETURN{$min\ $}
			\\
			\STATE{\hspace{5pt} Algorytm znajdowania nastepnika}
			\STATE{\textsc{Three-Successor}'(x)}
			\IF{$right[x]\ \neq NIL$}
				\item
				\RETURN{\textsc{Three-Minimum}$(right[x])\ $}
			\ENDIF
				\item $y \leftarrow p[x]$
			\WHILE{$y \neq NIL$ i $x= right[y]\ $}
				\FOR{$x \leftarrow y$}
					\item $y \leftarrow p[y]$
				\ENDFOR
			\ENDWHILE
			\RETURN{$y\ $}
			\\
			\STATE{\hspace{5pt} Wybor aktywnosci:}
			\STATE{\textsc{Greedy-Activity-Selector}(s,f)} \\
			$n \leftarrow length[s]\ $\\
			$A \leftarrow \{1\} $ \\
			$j \leftarrow 1 $
			\FOR{$i \leftarrow 2\ to\ n\ $}
				\IF{$s_{i} \geq f_{j}$}
					\item $A \leftarrow A \cup i$
					\item $j \leftarrow i $
				\ENDIF
			\ENDFOR
			\RETURN{$A\ $}
	\end{algorithmic}
	
	\begin{table}
		\centering
		\caption{Symbole w grafice zolwia}
		\label{tab:SymboleWGraficeZolwia}
			\begin{tabular}{|c|c|}
			\hline
				Symbol & Znaczenie \\ \hline
				F & idz do przodu jeden krok o dlugosci$ l\ $i narysuj linie od poprzedniej pozycji do nowej \\ \hline
				f & idz do przodu jeden krok o dlugosci$ l\ $ale nie rysuj linii \\ \hline
				+ & obroc sie w lewo (przeciwnie do ruchu wskazowek zegara) o staly kat $\delta $ \\ \hline
				- & obroc sie w prawo (zgodnie z ruchem wskazowek zegara) o staly kat $\delta $ \\ \hline
			\end{tabular}
	\end{table}

	\begin{table}
		\centering
		\caption{Interpretacja matematyczna symboli w grafice zolwia}
		\label{tab: InterpretacjaMatematycznaSymboliWGraficeZolwia}
			\begin{tabular}{|c|c|}
			\hline
				Symbol & stan (x,y,$\alpha)$ przechodzi w \\ \hline \hline
				F & $(x+lcos\alpha, y+lsin\alpha,\alpha)\ $ \\ \hline
				f & $(x+lcos\alpha, y+lsin\alpha,\alpha)\ $ \\ \hline
				+ & (x,y, $\alpha-\beta $) \\ \hline
				- & (x,y, $\alpha+\beta$) \\ \hline
			\end{tabular}
	\end{table}
	\begin{table}
		\centering
			\begin{tabular}{|l|l|}
			\hline
				\multicolumn{2}{|c|}{Komendy, ktore struja poruszaniem sie zolwia} \\ \hline \hline
				Symbol & Znaczenie \\ \hline
				f(l), G(l) & idz do przodu jeden krok o dlugosci $l\ $i naryzuj linie od poprzedniej pozycji do nowej \\ \hline
				f(l), g(l) & idz do porzodu o jeden krok o dlugosci $l\ $ale nie rysuj linii \\ \hline
				@O(r) & narysuj sfere o promieniu r i srodku w aktualnej pozycji \\ \hline
				\multicolumn{2}{|c|}{Komendy, ktore steruja obrotami zolwia} \\ \hline \hline
				+($\delta$) & obroc sie w lewo o kat $\delta$ do okolo osi $\overrightarrow{U}$. Macierz rotacji $R_{U}(\delta)$. \\ \hline
				-($\delta$) & obroc sie w prawo o kat $\delta$ do okolo osi $\overrightarrow{U}$. Macierz rotacji $R_{U}(-\delta)$. \\ \hline
				$\&(\delta)$ & pochyl sie o kat $\delta$ dokola osi $\overrightarrow{L}$. Macierz rotacji $R_{L}(\delta)$ \\ \hline
				$^\frown(\delta)$ & przenies sie o kat $\delta$ dokola $\overrightarrow{L}$. Macierz rotacji $R_{L}(-\delta)$ \\ \hline
				$/$ & przewroc sie na lewy bok o kat $\delta$ dokola osi $\overrightarrow{H}$. Macierz rotacji $R_{H}(\delta)$ \\ \hline
				$\text{\ }$ & przewroc sie na prawy bok o kat $\delta$ dokola osi $\overrightarrow{H}$. Macierz rotacji $R_{H}(-\delta)$ \\ \hline
				$\|$ & obroc o $180^o$ dokola osi $\overrightarrow{U}$, ten sam efekt mozna uzyskac uzywajac komend +(180) lub -(180) \\ \hline
			\end{tabular}
		\caption{Grafika zolwia-symbole dodatkowe}
		\label{GrafikaZolwiaSymboleDodatkowe}
	\end{table}	
	
	\begin{table}
		\centering
			\begin{tabular}{|l|c|c|c|c|c|c|c|}
				\hline
				odleglosci miedzy stopniami & \multicolumn{7}{|c|}{I--II--II--IV--V--VI--VII--VIII}\\ \hline \hline
				dur naturalny & 1 & 1 & $\frac{1}{2}$ & 1 & 1 & 1 & $\frac{1}{2}$ \\ \hline
				dur harmonicznze & 1 & 1 & $\frac{1}{2}$ & 1 & $\frac{1}{2}$ & $1\frac{1}{2}$ & $\frac{1}{2}$ \\ \hline
				dur miekkie & 1 & 1 & $\frac{1}{2}$ & 1 & $\frac{1}{2}$ & 1 & 1 \\ \hline
				moll naturalne & 1 & $\frac{1}{2}$ & 1 & 1 & $\frac{1}{2}$ & 1 & 1 \\ \hline
				moll harmoniczne & 1 & $\frac{1}{2}$ & 1 & 1 & $\frac{1}{2}$ & $1\frac{1}{2}$ & $\frac{1}{2}$ \\ \hline
				moll doryckie & 1 & $\frac{1}{2}$ & 1 & 1 & 1 & 1 & $\frac{1}{2}$ \\ \hline
			\end{tabular}
		\caption{Skale muzyczne i odleglosci miedzy poszczegolnymi stopniami pomiedzy nimi}
		\label{tab: skale}
	\end{table}
	
	\begin{table}
		\centering
			\begin{tabular}{|l|l|}
			\hline
				Ciag symboli & Znaczenie \\ \hline \hline
				$(CGC)\rightarrow(GBC)$ & kontekst polifoniczny: jesli biezace nuty (biezace symbole) to CGC zagraj (przepisz) jednoczesnie GBC \\ \hline
				$CE$---$G\rightarrow D$ & czasowy kontekst: jesli biezaca nuta to G, a poprzedzaja ja C i E, to zagraj nute D \\ \hline
				$(CE)$---$(GC)\rightarrow D(CE)$ & polifoniczny i czasowy kontekst: jesli biezace nuty grane jednoczesnie to GC i poprzedzone sa nutami CE granymi jednoczesnie to zagraj D nastepujace po C i E \\ \hline
			\end{tabular}
		\caption{Przyklad zastosowanie L-systemow do dzwiekow muzycznych}
		\label{tab: dzwieki}
	\end{table}
	
	\begin{table}
		\centering
		\caption{Tymczasowe Polskie Normy Zywienia-fragment}
		\label{tab: NormyZywienia}
			\begin{tabular}{l|c|c|c|c|c}
				\hline
					\textbf{Grupy ludnosci} & \multicolumn{2}{c}{\textbf{Wartosc}} & \multicolumn{2}{|c|}{\textbf{Bialka}} & \multirow{2}{*}{\textbf{Tluszcze}} \\ \cline{4-5}
					& \multicolumn{2}{|c|}{\textbf{energetyczna}} & \textbf{ogolem} & \textbf{zwierzece} & \\ \cline{2-6}
					& kcal & kJ & g & g & g \\ \hline
					\textit{Mezczyzni:} & & & & & \\ 
					zajecia siedzace & 2600 & 10900 & 75 & 25-50 & 75-100 \\	
					umiarkowana praca & 3200 & 13400 & 55 & 28-55 & 90-125 \\					 ciezka praca & 4000 & 16700 & 95 & 30-60 & 110-155 \\
					bardzo ciezka praca & 4500 & 18800 & 100 & 35-65 & 125 - 165 \\ \hline
					\textit{Kobiety:} & & & & & \\
					zajecia siedziace & 2300 & 9600 & 70 & 23-45 & 65-90 \\
					umiarkowana praca & 2800 & 11700 & 80 & 27-53 & 80-110 \\
					ciezka praca & 3200 & 13400 & 90 & 30-40 & 90-120 \\
					ciezarne & 2800 & 11700 & 95 & 35-45 & 80-100 \\
					karmiace & 3400 & 14200 & 110 & 35-50 & 95-130 \\ \hline
			\end{tabular}
		%\vspace{-0.2cm}
		%\newline
	\caption*{\footnotesize{\textit{Zrodlo: J. Piekarska, A. Szczygiel, M. Los-Kuczera, Popularne tabele wartości odżywczych żywności. Cyt. za. Praca zbiorowa, Kuchnia Polska, Państwowe Wydawnictwo Ekonomiczne, Warszawa 1990}}}
	\end{table}
	
	\begin{figure}[h]
		\centering
			\includegraphics{buzia.jpg}
		\caption{Uśmiechnięta buzia bez zmian}
	\end{figure}
	\begin{figure}[h]
		\centering
			\includegraphics{buzia.jpg}
			\includegraphics{buzia.jpg}
		\caption{Czy na lewo czy na prawo do uśmiechu przystąp żwawo}
	\end{figure}
	\begin{figure}[h]
		\centering
			\includegraphics[scale=0.4]{buzia.jpg}
			\includegraphics[scale=0.3]{buzia.jpg}
			\includegraphics[scale=0.2]{buzia.jpg}
			\includegraphics[scale=0.1]{buzia.jpg}
		\caption{Podaj uśmiech dalej - a rozprzestrzeni się na innych}
	\end{figure}
	
\end{document}