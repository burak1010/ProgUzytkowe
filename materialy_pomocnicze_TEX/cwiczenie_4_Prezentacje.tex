\documentclass[]{beamer}
%\usepackage[MeX]{polski}
\usepackage[cp1250]{inputenc}
\usepackage{polski}
%\usepackage[utf8]{inputenc}
\beamersetaveragebackground{blue!10} % {kolor slajdu!przezroczystkosc_koloru}
\usetheme{Warsaw}
\usecolortheme[rgb={0.1,0.5,0.7}]{structure}
\usepackage{beamerthemesplit}
\usepackage{multirow}
\usepackage{multicol}
\usepackage{array}
\usepackage{graphicx}
\usepackage{enumerate}
\usepackage{amsmath} %pakiet matematyczny
\usepackage{amssymb} %pakiet dodatkowych symboli
\usepackage{wrapfig}
	\title{Materia�y do kolokwium z \TeX 'a}
\begin{document}
	\frame{
		\frametitle{Formatowanie tekstu}
		\begin{block}{Klasy dokument�w w \TeX}
			\begin{table}
				\begin{tabular}{l|l}
					Klasy & Podzial dokumentu w formacie article \\
					\begin{enumerate}\item article-artyku?y \end{enumerate} & a 
				\end{tabular}
			\end{table}
		\end{block}
		\begin{block}{Punktowanie i numeracja}
			\begin{table}
				\centering
				\begin{tabular}{l l}
					Punktowanie & Numeracja \\ \hline \\
					$\backslash$begin\{itemize\} & $\backslash$begin\{enumerate\} \\
						\hspace{20pt} $\backslash$item punkt 1 & \hspace{20pt} $\backslash$item punkt 1 \\
						\hspace{20pt} $\backslash$item punkt 2 & \hspace{20pt} $\backslash$item punkt 2 \\
					$\backslash$end\{itemize\} & $\backslash$end\{enumerate\}
				\end{tabular}
			\end{table}
		\end{block}
	}
	\frame{
		\frametitle{Formuy? matematyczne}
		\begin{block}{Ka�d� formu�� zapisuj w znacznikach}
		 $\backslash$begin\{equation\}\\
				%\begin{center} 
				\indent Tu wstaw jaka� formu�� matematyczn�.\indent \\ %\end{center}
		 $\backslash$end\{equation\}
		\end{block}
		\begin{exampleblock}{Przyk�adowe formu�y matematyczne}
			\begin{enumerate}
				\item U�amek: $\backslash$frac\{1\}\{x\}
				%\item Indeks g�rny: x\^  \{2\} %=$$x^{2}$$ \\
				\item Indeks dolny: x\_2 %$$= x_2$$
				\item Pierwiastek z x: $\backslash$sqrt\{x\} %$$= \sqrt{x} $$
				\item Pierwiastek z x stopnia y: $\backslash$sqrt[y]\{x\} %=$$\sqrt[y]{x}$$ \\
			\end{enumerate}
		\end{exampleblock}
		\begin{alertblock}{Uwaga!}
			Znaki specjalne typu \o,  \&, \%, \$, \{ etc. musz? by? uprzedzane znakiem "backslash", np. $\backslash$\& .
		\end{alertblock}
		}
	\frame{
		\frametitle{Tabele i obrazy}
	}
	\frame{
		\frametitle{Beamer}
	}
\end{document}