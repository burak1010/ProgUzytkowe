\documentclass[12pt]{article}
\usepackage[a4paper,left=3cm,right=2.5cm,top=4cm,bottom=2.5cm]{geometry}
\usepackage[MeX]{polski}
%\usepackage[cp1250]{inputenc}
\usepackage{polski}
%\usepackage[utf8]{inputenc}
\usepackage[pdftex, hidelinks=true, colorlinks=true, allcolors=blue]{hyperref}
\usepackage{makeidx}
\usepackage[tableposition=top]{caption}
\usepackage{algorithmic}
\usepackage{graphicx}
\usepackage{enumerate}
\usepackage{multirow}
\usepackage{amsmath} %pakiet matematyczny
\usepackage{amssymb} %pakiet dodatkowych symboli
\usepackage{xcolor}
\begin{document}
\fontfamily{qtm}
\noindent\textbf{Arguing for Wisdom in the University: An Intellectual Autobiography \\
Nicholas Maxwell \\
University College London \\}
\href{mailto:nicholas.maxwell@ucl.ac.uk}{\underline{nicholas.maxwell@ucl.ac.uk}} \\
Published in \textit{Philosophia} vol. 40, no, 4, pp. 663-704, 2012
\begin{flushleft}
\textbf{The Key to Wisdom} \\
\hspace{10pt}Nearly forty years ago I discovered a profoundly significant idea – or so I believe. Since then, I have expounded and developed the idea in six books$^{1}$ and countless articles published in academic journals and other books.$^{2}$ I have talked about the idea in universities and at conferences all over the UK, in Europe, the USA, Canada, and Taiwan. And yet, alas, despite all this effort, few indeed are those who have even heard of the idea. I have not even managed to communicate the idea to my fellow philosophers.$^{3}$ \\
\hspace{10pt}What did I discover? Quite simply: the key to wisdom.$^{4}$ For over two and a half thousand years, philosophy (which means “love of wisdom”) has sought in vain to discover how humanity might learn to become wise – how we might learn to create an enlightened world. For the ancient Greek philosophers, Socrates, Plato and the rest, discovering how to become wise was the fundamental task for philosophy. In the modern period, this central, ancient quest has been laid somewhat to rest, not because it is no longer thought important, but rather because the quest is seen as unattainable. The record of savagery and horror of the last century is so extreme and terrible that the search for wisdom, more important than ever, has come to seem hopeless, a quixotic fantasy. Nevertheless, it is this ancient, fundamental problem, lying at the heart of philosophy, at the heart, indeed, of all of thought, morality, politics and life, that I have solved. Or so I believe.
\hspace{10pt}When I say I have discovered the key to wisdom, I should say, more precisely, that I have discovered the \textit{methodological} key to wisdom. Or perhaps, more modestly, I should say that I have discovered that science contains, locked up in its astounding success in acquiring knowledge and understanding of the universe, the methodological key to wisdom. I have discovered a recipe for creating a kind of organized inquiry rationally designed and devoted to helping humanity learn wisdom, learn to create a more enlightened world.\\
\hspace{10pt}What we have is a long tradition of inquiry – extraordinarily successful in its own terms – devoted to acquiring knowledge and technological know-how. It is this that has created the modern world, or at least made it possible. But scientific knowledge and technological know-how are ambiguous blessings, as more and more people, these days, are beginning to recognize. They do not guarantee happiness. Scientific knowledge and technological know-how enormously increase our power to act. In endless ways, this vast increase in our power to act has been used for the public good – in health, agriculture, transport, communications, and countless other ways. But equally, this enhanced power to act can be used to cause human harm, whether unintentionally, as in environmental damage (at least initially), or intentionally, as in war. It is hardly too much to say that all our current global problems have come about because of science and technology. The appalling destructiveness of modern warfare and terrorism, vast inequalities in wealth and standards of living between first and third worlds, rapid population growth, environmental damage – destruction of tropical rain forests, rapid extinction of species, global warming, pollution of sea, earth and air, depletion of finite natural resources – all only exist today because of modern science and technology. Science and technology lead to modern industry and agriculture, to modern medicine and hygiene, and thus in turn to population growth, to modern armaments, conventional, chemical, biological and nuclear, to destruction of natural habitats, extinction of species, pollution, and to immense inequalities of wealth around the globe.
\end{flushleft}
\end{document}