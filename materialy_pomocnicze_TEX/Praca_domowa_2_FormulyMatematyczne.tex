\documentclass{article}
\usepackage[a4paper,left=3.5cm,right=2.5cm,top=2.5cm,bottom=2.5cm]{geometry}
%%\usepackage[MeX]{polski}
%%\usepackage[cp1250]{inputenc}
\usepackage{polski}
\usepackage[utf8]{inputenc}
\usepackage[pdftex]{hyperref}
\usepackage{makeidx}
\usepackage[tableposition=top]{caption}
\usepackage{algorithmic}
\usepackage{graphicx}
\usepackage{enumerate}
\usepackage{multirow}
\usepackage{amsmath} %pakiet matematyczny
\usepackage{amssymb} %pakiet dodatkowych symboli
\begin{document}
\section{Formuły matematyczne}
	Plik c2-PU\_new.pdf, formuły: 11, 16, 26, 28.
	\newline
	\begin{equation}
		\lim_{n\to\infty}\sum_{k=1}^{n}\frac{1}{k^{2}}=\frac{\pi^{2}}{6} 
	\end{equation}
	\begin{equation}
		[x]_A =\{ y\in\mathbb{U}:a(x)=a(y),\forall a\in\mathbb{A} \} \text{, where the central object }x\in\mathbb{U}
	\end{equation}
		\begin{equation}
			P\bigg(A=2 \bigg|\frac{A^{2}}{B}>4\bigg)
	\end{equation}
\begin{equation}
			c^{\prime}_{ij} = 
			\left\{
				\begin{array}[c]
					c_{ij} \text{ gdy }d(x_i)\neq d(x_j)\\
					\phi \text{ gdy }d(x_i) = d(x_j) \\
				\end{array}
			\right.
	\end{equation}
\end{document}