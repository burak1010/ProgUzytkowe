\documentclass[]{beamer}
%\usepackage[MeX]{polski}
\usepackage[cp1250]{inputenc}
\usepackage{polski}
%\usepackage[utf8]{inputenc}
\beamersetaveragebackground{blue!10} % {kolor slajdu!przezroczystkosc_koloru}
\usetheme{Warsaw}
\usecolortheme[rgb={0.1,0.5,0.7}]{structure}
\usepackage{beamerthemesplit}
\usepackage{multirow}
\usepackage{multicol}
\usepackage{array}
\usepackage{graphicx}
\usepackage{enumerate}
\usepackage{amsmath} %pakiet matematyczny
\usepackage{amssymb} %pakiet dodatkowych symboli
	\title{Materia�y do kolokwium z \TeX 'a}
\begin{document}
	\frame{
		\frametitle{Formatowanie tekstu}
		%\begin{block}{Blok}
		%	Tu podaj tre��.
		%\end{block}
		%\begin{exampleblock}{Blok do podawania przyk�ad�w}
		%	Tu podaj przyk�ad.
		%\end{exampleblock}
		%\begin{alertblock}{Blok uwag.}
		%	Tu podaj istotn� uwage.
		%\end{alertblock}
	}
	\frame{
		\frametitle{Formu�y matematyczne}
		\begin{block}{Ka�d� formu�� zapisuj w znacznikach}
		 \ begin\{equation\} \\
				Tu wstaw jaka� formu�� matematyczn�. \\
		 \ end\{equation\}
		\end{block}
		\begin{exampleblock}{Przyk�adowe formu�y matematyczne}
			\begin{enumerate}
				\item U�amek: frac\{1\}\{x\}$$= \frac{1}{x}$$
				\item Indeks g�rny: x\^\{2\}$$= x^{2}$$
				%%\item Indeks dolny: x\_2 $$= x_2$$
				%%\item Pierwiastek z x: sqrt\{x\} $$= \sqrt{x} $$
				\item Pierwiastek sqrt[y]\{x\} = $$\sqrt[y]{x}$$
			\end{enumerate}
		\end{exampleblock}
	}
	\frame{
		\frametitle{Tabele i obrazy}
	}
	\frame{
		\frametitle{Beamer}
	}
\end{document}