\documentclass{article}
\usepackage[a4paper,left=3.5cm,right=2.5cm,top=2.5cm,bottom=2.5cm]{geometry}
%%\usepackage[MeX]{polski}
\usepackage[cp1250]{inputenc}
\usepackage{polski}
%%\usepackage[utf8]{inputenc}
\usepackage[pdftex]{hyperref}
\usepackage{makeidx}
\usepackage[tableposition=top]{caption}
\usepackage{algorithmic}
\usepackage{graphicx}
\usepackage{enumerate}
\usepackage{multirow}
\usepackage{amsmath} %pakiet matematyczny
\usepackage{amssymb} %pakiet dodatkowych symboli
\begin{document}
	\begin{equation}
		|U|=sup\{|x-y|:x,y \in U\}
	\label{eq: 1}
	\end{equation}
	\begin{equation}
		F \subset \sum_{i=1}^{\infty} U_{i} \text{, gdzie } 0 <|U_{i}|\leqslant \delta \text{ dla ka�dego i,}
	\label{eq: 2}
	\end{equation}
	\begin{equation}
		H^8_{\partial} = inf \Bigg\{ \sum_{i=1}^{\infty}|U_i|^{s} :\{U_i\} \text{ jest } \delta \text{ - pokryciem F} \Bigg\}
	\label{eq: 3}
	\end{equation}
	\begin{equation}
		H^{s}(F) = \lim_{\delta\to 0} H_\delta^{s}(F)
	\label{eq: 4}
	\end{equation}
	\begin{equation}
		H^{s}(F) =  
			\left\{
				\begin{array}[c]
					c{\infty} \text{ je�li }s < dim_H F\\
					0 \text{ je�li } s > dim_H F \\
				\end{array}
			\right.
	\label{eq: 5}
	\end{equation}
	\begin{equation}
		\overline{dim}_B F = lim_{\delta \to 0} sup \frac{log N_\delta (F)}{-log \delta}
	\label{eq: 6}
	\end{equation}
	\begin{equation}
		\begin{array}[c]
			c{\overline{dim}_B F = \lim_{\delta \to 0} sup \frac{log N_\delta (F)}{-log \delta} \leqslant \lim_{k \to \infty} sup \frac{log2^{k}}{log3^{k-1}} = \frac{log2}{log3}} \\ \\
			\phi_t[\phi_\tau (x)] = \phi_{t+\tau} (x) \\ \\
			\frac{dx}{dt} = -10x + 10y \\
			\frac{dy}{dt} = rx - y - xz \\
			\frac{dz}{dt} = \frac{8}{3}z + xy	\hspace{16.5pt}
		\end{array}
	\label{eq: 7}
	\end{equation}
	\begin{equation}
		\begin{array}[c]
			c{\rho(f(x_1), f(x_2)) \leqslant \lambda \rho(x_1, x_2)} \\ \\
			R_U (\delta) = 
			\left[ 
					\begin{array}[c]
						r{cos\delta \ sin\delta \ 0} \\
						-sin\delta \ cos\delta \ 0 \\
						0 \hspace{15pt} \ 0 \hspace{10pt} \  1
					\end{array}
			\right]
		\end{array}	
	\label{eq: 8}
	\end{equation}
	
		\begin{verbatim}
	for{int i=0;i<10;i++}
	{
	cout<<"i="<<i;
	}
	\end{verbatim}
	Algorytm zach�anny 
	\indent GREEDY(M,w)
	\indent uporz�dkuj S[M] nierosn�co wed�ug wagi w 
	\begin{algorithmic}
	\FOR x \in S[M], brane w porz�dku nierosn�cym wed�ug wagi w(x) \DO
		\item \IF A \cup x \in \rho[M] \THEN \\
			A \gets A \cup x
		\ENDIF
	\ENDFOR
	\RETURN A
	\end{algorithmic}
\end{document}